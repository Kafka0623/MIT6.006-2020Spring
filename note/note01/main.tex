%!TeX program = xelatex
\documentclass[10.5pt,hyperref,a4paper,UTF8]{ctexart}
\usepackage{RUCReport}
\usepackage{listings}
\usepackage{xcolor}
\usepackage{threeparttable}
\usepackage{booktabs}
\usepackage{array}
\usepackage{subcaption}
\usepackage{graphicx}
\usepackage{geometry}
\usepackage{float}
\geometry{a4paper, margin=1in}
\renewcommand\thesubfigure{\arabic{subfigure}} % 子图编号改为中文小写
% 定义可能使用到的颜色
\definecolor{CPPLight}  {HTML} {686868}
\definecolor{CPPSteel}  {HTML} {888888}
\definecolor{CPPDark}   {HTML} {262626}
\definecolor{CPPBlue}   {HTML} {4172A3}
\definecolor{CPPGreen}  {HTML} {487818}
\definecolor{CPPBrown}  {HTML} {A07040}
\definecolor{CPPRed}    {HTML} {AD4D3A}
\definecolor{CPPViolet} {HTML} {7040A0}
\definecolor{CPPGray}  {HTML} {B8B8B8}
\definecolor{keywordcolor}{rgb}{0.8,0.1,0.5}
\definecolor{webgreen}{rgb}{0,.5,0}
\definecolor{bgcolor}{rgb}{0.92,0.92,0.92}

\makeatletter
\renewcommand\@cite[2]{\textsuperscript{[#1]}} % 将引用变为右上角的形式
\makeatother

\hypersetup{
    citecolor=black,  % 修改文献引用的颜色为黑色
}

\lstset{
    breaklines = true,                                   % 自动将长的代码行换行排版
    extendedchars=false,                                 % 解决代码跨页时,章节标题,页眉等汉字不显示的问题
    columns=fixed,       
    numbers=left,                                        % 在左侧显示行号
    basicstyle=\zihao{-5}\ttfamily,
    numberstyle=\small,
    frame=none,                                          % 不显示背景边框
    % backgroundcolor=\color[RGB]{245,245,244},            % 设定背景颜色
    keywordstyle=\color[RGB]{40,40,255},                 % 设定关键字颜色
    numberstyle=\footnotesize\color{darkgray},           % 设定行号格式
    commentstyle=\it\color[RGB]{0,96,96},                % 设置代码注释的格式
    stringstyle=\rmfamily\slshape\color[RGB]{128,0,0},   % 设置字符串格式
    showstringspaces=false,                              % 不显示字符串中的空格
    % frame=leftline,topline,rightline, bottomline         %分别对应只在左侧,上方,右侧,下方有竖线
    frame=shadowbox,                                     % 设置阴影
    rulesepcolor=\color{red!20!green!20!blue!20},        % 阴影颜色
    basewidth=0.6em,
}

\lstdefinestyle{CPP}{
    language=c++,                                        % 设置语言
    morekeywords={alignas,continute,friend,register,true,alignof,decltype,goto,
    reinterpret_cast,try,asm,defult,if,return,typedef,auto,delete,inline,short,
    typeid,bool,do,int,signed,typename,break,double,long,sizeof,union,case,
    dynamic_cast,mutable,static,unsigned,catch,else,namespace,static_assert,using,
    char,enum,new,static_cast,virtual,char16_t,char32_t,explict,noexcept,struct,
    void,export,nullptr,switch,volatile,class,extern,operator,template,wchar_t,
    const,false,private,this,while,constexpr,float,protected,thread_local,
    const_cast,for,public,throw,std},
    emph={map,set,multimap,multiset,unordered_map,unordered_set,
    unordered_multiset,unordered_multimap,vector,string,list,deque,
    array,stack,forwared_list,iostream,memory,shared_ptr,unique_ptr,
    random,bitset,ostream,istream,cout,cin,endl,move,default_random_engine,
    uniform_int_distribution,iterator,algorithm,functional,bing,numeric,},
    emphstyle=\color{CPPViolet}, 
}

\lstdefinestyle{Java}{
    language=[AspectJ]Java,
    keywordstyle=\color{keywordcolor}\bfseries
}

\lstdefinestyle{Python}{
    language=Python,
}


%%-------------------------------正文开始---------------------------%%
\begin{document}

%%-----------------------封面--------------------%%
\cover

%%------------------摘要-------------%%
\begin{abstract}
%
XXXXX\\
\textbf{关键词:}XXX
%
\end{abstract}

%%--------------------------目录页------------------------%%
\newpage
\tableofcontents
\thispagestyle{empty} % 目录不显示页码

%%------------------------正文页从这里开始-------------------%
\newpage
\setcounter{page}{1} % 让页码从正文开始编号

%%可选择这里也放一个标题
\begin{center}
    \title{ \Huge \textbf{XXX} }
\end{center}


\thispagestyle{empty} % 首页不显示页码


\section{goal}
\subsection{solve computational problems}
解决计算问题
\subsection{prove correctness}
证明算法的正确性    
\subsection{argue efficiency}
论证算法的效率
Don't measure time, measure the number of operations.
Use asymptotic analysis(渐进分析), which $O(n)$ means the upper bounds, $\Omega(n)$ means the lower bounds, and $\Theta(n)$ for both.
O Notation: Non-negative function(非负函数) $g(n)$is in $O(f(n))$ if and only if there exists a positive real number $c$ and positive integer $n_0$ such that $g(n) \leq c \cdot f(n)$, for all $n > n_0$.
$\Omega$ Notation: Non-negative function $g(n)$ is in $\Omega(f(n))$ 
if and only if there exists a positive real number $c$ and positive integer $n_0$ 
such that 
\[
c \cdot f(n) \leq g(n) \quad \text{for all } n \geq n_0.
\]
$\Theta$ Notation: Non-negative function $g(n)$ is in $\Theta(f(n))$ if and only if $g(n) \in O(f(n)) \cap \Omega(f(n))$.


\section{What is a computational problem?}
Some kind of predict, say that we can check. If given a input and output, we can check whether the output is correct for the input.

\section{What is an algorithm?}
some kind of functions that takes these inputs, maps(映射) it to a single output, and that output better be correct based on the problem.
\section{RAM}
Random access memory(RAM) means that we can randomly access different places in memory in constant time.(RAM 提供了常数时间的随机访问能力,使得我们可以直接跳到任意内存地址,而不需要顺序扫描)

%在reference.bib文件中填写参考文献,此处自动生成
\newpage
\reference


\end{document}