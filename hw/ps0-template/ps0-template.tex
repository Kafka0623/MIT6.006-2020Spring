%
% 6.006 problem set 0 solutions template
%
\documentclass[12pt,twoside]{article}

\usepackage{mathptmx} % Use Times font for text and math
\usepackage{amsmath} % For align and split environments
\input{macros-sp20}
\newcommand{\theproblemsetnum}{0}

\title{6.006 Problem Set 0}

\begin{document}

\handout{Problem Set \theproblemsetnum}
\setlength{\parindent}{0pt}
\medskip\hrulefill\medskip

{\bf Name:} Xinzhe Cui

\medskip\hrulefill

%%%%%%%%%%%%%%%%%%%%%%%%%%%%%%%%%%%%%%%%%%%%%%%%%%%%%
% See below for common and useful latex constructs. %
%%%%%%%%%%%%%%%%%%%%%%%%%%%%%%%%%%%%%%%%%%%%%%%%%%%%%

% Some useful commands:
% $f(x) = \Theta(x)$
% $T(x, y) \leq \log(x) + 2^y + \binom{2n}{n}$
% \ttt{code\_function}


% You can create unnumbered lists as follows:
% \begin{itemize}
%     \item First item in a list
%         \begin{itemize}
%             \item First item in a list
%                 \begin{itemize}
%                     \item First item in a list
%                     \item Second item in a list
%                 \end{itemize}
%             \item Second item in a list
%         \end{itemize}
%     \item Second item in a list
% \end{itemize}

% You can create numbered lists as follows:
% \begin{enumerate}
%     \item First item in a list
%     \item Second item in a list
%     \item Third item in a list
% \end{enumerate}

% You can write aligned equations as follows:
% \begin{align}
%     \begin{split}
%         (x+y)^3 &= (x+y)^2(x+y) \\
%                 &= (x^2+2xy+y^2)(x+y) \\
%                 &= (x^3+2x^2y+xy^2) + (x^2y+2xy^2+y^3) \\
%                 &= x^3+3x^2y+3xy^2+y^3
%     \end{split}
% \end{align}

% You can create grids/matrices as follows:
% \begin{align}
%     A =
%     \begin{bmatrix}
%         A_{11} & A_{21} \\
%         A_{21} & A_{22}
%     \end{bmatrix}
% \end{align}

\begin{problems}

\problem  % Problem 1
$A = \{1,6,12,13,9\}$
$B = \{3,6,12,15\}$
\begin{problemparts}
\problempart % Problem 1a
$A \cap B = \{6,12\}$
\problempart % Problem 1b
$|A \cup B| = 7$
\problempart % Problem 1c
$|A - B| = 3$
\end{problemparts}

\problem  % Problem 2
\begin{problemparts}
\problempart % Problem 2a
$E(X)=\frac{1}{2} \times 3 =\frac{2}{3}$
\problempart % Problem 2b
$E(Y)=E(Y_1)\times E(Y_2)=(\frac{1}{6} \times \sum\limits_{i=1}^{6} i )^2= \frac{49}{4}$
\problempart % Problem 2c
$E(X+Y)=E(X)+E(Y)=\frac{2}{3}+\frac{49}{4}=\frac{155}{12}$
\end{problemparts}

\problem  % Problem 3
$A = 100$
$B = 18$
\begin{problemparts}
\problempart % Problem 3a
$A \equiv B \pmod{2}$ $True$
\problempart % Problem 3b
$A \equiv B \pmod{3}$ $False$
\problempart % Problem 3c
$A \equiv B \pmod{4}$ $False$
\end{problemparts}

\problem  % Problem 4
We prove by induction that
\[
\sum_{i=1}^{n} i^3 = \left( \frac{n(n+1)}{2} \right)^2 .
\]

\textbf{Base case:} When $n=1$,
\[
\sum_{i=1}^{1} i^3 = 1, \quad \left(\frac{1(1+1)}{2}\right)^2 = 1.
\]
Thus the formula holds for $n=1$.

\textbf{Inductive step:} Assume that
\[
\sum_{i=1}^{n} i^3 = \left(\frac{n(n+1)}{2}\right)^2
\]
holds for $n=k$. Then for $n=k+1$,
\[
\sum_{i=1}^{k+1} i^3
= \sum_{i=1}^{k} i^3 + (k+1)^3
= \left(\frac{k(k+1)}{2}\right)^2 + (k+1)^3
= \left(\frac{(k+1)(k+2)}{2}\right)^2 .
\]

Therefore, by induction, the formula holds for all $n \geq 1$. \qed

\problem  % Problem 5
\begin{proof}
We prove by induction on the number of vertices $|V|$.  

\textbf{Base case:}  
When $|V|=1$, the graph has no edges ($|E|=0$). Clearly, it is acyclic.  

\textbf{Inductive hypothesis:}  
Assume that for some $k \geq 1$, every connected undirected graph with $|V|=k$ and $|E|=k-1$ is acyclic.  

\textbf{Inductive step:}  
Consider a connected graph $G=(V,E)$ with $|V|=k+1$ and $|E|=k$.  
Since $|E|=|V|-1$, there must exist at least one vertex of degree $1$ (a leaf).  
Remove this vertex $v$ and its incident edge.  
The remaining graph has $k$ vertices and $k-1$ edges, and it is still connected.  
By the inductive hypothesis, this smaller graph is acyclic.  
Adding back vertex $v$ and its edge cannot create a cycle, because $v$ had degree $1$.  
Thus, $G$ is acyclic.  

\textbf{Conclusion:}  
By induction, every connected undirected graph $G$ with $|E| = |V|-1$ is acyclic.  
\end{proof}

\vfill
\problem  % Problem 6


Submit your implementation to {\small\url{alg.mit.edu}}.

\begin{lstlisting}
def count_long_subarray(A):
    '''
    Input:  A     | Python Tuple of positive integers
    Output: count | number of longest increasing subarrays of A
    '''
    count = 0
    ##################
    # YOUR CODE HERE #
    ##################
    n = len(A)
    if n == 0:
        return 0

    max_len = 1      # 当前已知的最长递增子数组长度
    count = 0        # 达到 max_len 的段数
    curr = 1         # 当前递增段长度

    for i in range(1, n):
        if A[i] > A[i-1]:
            curr += 1
        else:
            # 递增段在 i-1 处结束,结算这一段
            if curr > max_len:
                max_len = curr
                count = 1
            elif curr == max_len:
                count += 1
            curr = 1  # 重新开始新的一段

    # 循环结束,结算最后一段
    if curr > max_len:
        max_len = curr
        count = 1
    elif curr == max_len:
        count += 1

    return count

\end{lstlisting}

\end{problems}

\end{document}
